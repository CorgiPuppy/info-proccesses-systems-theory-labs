\documentclass[12pt, a4paper]{report}
\usepackage[top=1cm, left=1cm, right=1cm]{geometry}

\usepackage[utf8]{inputenc}
\usepackage[russian]{babel}

\usepackage{array}
\newcolumntype{M}[1]{>{\centering\arraybackslash}m{#1}}

\usepackage{hyperref}
\hypersetup{
	colorlinks,
	citecolor=black,
	filecolor=black,
	linkcolor=black,
	urlcolor=black
}

\usepackage{sectsty}
\allsectionsfont{\centering}

\usepackage{indentfirst}
\setlength\parindent{24pt}

\usepackage{makecell}

\usepackage{amsmath}

\usepackage{tikz}
\usetikzlibrary{shapes.geometric, arrows.meta, patterns}

\usepackage{listings}
\usepackage{xcolor}
\definecolor{codegreen}{rgb}{0,0.6,0}
\definecolor{codegray}{rgb}{0.5,0.5,0.5}
\definecolor{codepurple}{rgb}{0.58,0,0.82}
\definecolor{backcolour}{rgb}{0.95,0.95,0.92}
\lstdefinestyle{mystyle}{
    backgroundcolor=\color{backcolour},
    commentstyle=\color{codegreen},
    keywordstyle=\color{magenta},
    numberstyle=\normalsize\color{codegray},
    stringstyle=\color{codepurple},
    basicstyle=\ttfamily\footnotesize,
    breakatwhitespace=false,
    breaklines=true,
    captionpos=b,
    keepspaces=true,
    numbers=left,
    numbersep=5pt,
    showspaces=false,
    showstringspaces=false,
    showtabs=false,
    tabsize=2
}

\usepackage{graphicx}
\graphicspath{ {plots/pictures/} }

\usepackage{booktabs}
\usepackage{csvsimple}
\usepackage{longtable}
\usepackage{caption}

\usepackage{pgfplots}
\pgfplotsset{compat=newest, ticks=none}

\usepackage{wrapfig}
\usepackage{float}

\usepackage{multirow, tabularx}
\newcolumntype{Y}{>{\centering\arraybackslash}X}
\renewcommand{\arraystretch}{1.2}

\begin{document}
	\begin{titlepage}
		\begin{center}
			\large \textbf{Министерство науки и высшего образования Российской Федерации} \\
			\large \textbf{Федеральное государственное бюджетное образовательное учреждение высшего образования} \\
			\large \textbf{«Российский химико-технологический университет имени Д.И. Менделеева»} \\

			\vspace*{6cm}
			\LARGE \textbf{ЛЕКЦИЯ №2}

			\vspace*{4cm}
			\begin{flushright}
				\Large
				\begin{tabular}{>{\raggedleft\arraybackslash}p{9cm} p{10cm}}
					Выполнил студент группы КС-36: & Золотухин Андрей Александрович \\
					Ссылка на репозиторий: & https://github.com/ \\
					& CorgiPuppy/ \\
					& chem-tech-control-sys-labs \\
				\end{tabular}
			\end{flushright}

			\vspace*{5cm}
			\Large \textbf{Москва \\ 2025}
		\end{center}
	\end{titlepage}

	\tableofcontents
	\thispagestyle{empty}
	\newpage

	\pagenumbering{arabic}

	\section*{Термодинамический анализ}
	\addcontentsline{toc}{section}{Термодинамический анализ}

	\subsection*{Диссипативная функция многофазной гетерогенной среды}
	\addcontentsline{toc}{subsection}{Диссипативная функция многофазной гетерогенной среды}
	\large
	Получу диссипативную функцию многофазной гетерогенной среды - \textit{производство энтропии}, - которая играет важную роль в термодинамике необратимых процессов. \par
	Пусть имеется объём $V$, где находятся две фазы с объёмами $V_{1}$ и $V_{2}$. Пусть смесь в целом неравновесна. Но в каждой фазе в малом локальном объёме существует локальное термодинамическое равновесие; тогда для каждой из фаз (в малом локальном объёме выполняется соотношение Гиббса:
	\begin{center}
		$\rho_{i} T_{i} \frac{d_{i} S_{i}}{dt} = \rho_{i} \frac{d_{i} u_{i}}{dt} - \frac{\alpha_{i} P_{i}}{\rho_{i}^{0}}\frac{d_{i} \rho_{i}}{dt} - \rho_{i} \mu_{ik} \frac{d C_{ik}}{dt}$,
	\end{center}
	где $S_{i}$ - удельная энтропия $i$-фазы; $\mu_{ik}$ - химический потенциал $k$-ого компонента в $i$-фазе. \par
	Энтропия всей системы имеет вид
	\begin{center}
		$\rho S = \rho_{1}S_{1} + \int_{0}^{R} \rho_{2}^{0} f r S_{2} dr$.
	\end{center}
	\par
	Изменение энтропии всей системы во времени имеет вид:
	\begin{center}
		$\rho \frac{dS}{dt} = \rho_{i} \frac{d_{i} S_{i}}{dt} + \int_{0}^{R} \rho_{2}^{0} f r \frac{d_{2} S_{2}}{dt} dr + \int_{0}^{R} \rho_{2}^{0} f \eta (S_{2} - S_{1}) dr$.
	\end{center}
	\par
	Подставим в данную формулу соотношения Гиббса для каждой из фаз, а в соответствующие соотношения Гиббса подставим уравнения изменения внутренних энергий и концетраций компонентов и, приведя подобные члены, получим
	\begin{center}
		$\rho \frac{dS}{dt} = \{-\nabla(\frac{\vec{q_{1}}}{T_{1}}) + \nabla(\frac{\mu_{1k} \vec{J_{1k}}}{T_{1}})\} + \{\frac{1}{T_{1}} \tau_{1}^{kl} e_{1}^{kl} + \frac{\vec{q_{1}}}{T_{1}} gradT_{1} + \vec{J_{1k}} grad \frac{\mu_{1k}}{T_{1}} + \int_{0}^{R} q_{12} (\frac{1}{T_{2}} - \frac{1}{T_{1}}) dr + \frac{1}{T_{1}} \int_{0}^{R} \rho_{2}^{0} f r \vec{f_{12}} (\vec{\nu_{1}} - \vec{\nu_{2}}) dr + \int_{0}^{R} \rho_{2}^{0} f \eta [(\frac{\mu_{1k}}{T_{2}} - \frac{\mu_{2k}}{T_{2}}) + i_{1}(\frac{1}{T_{2}} - \frac{1}{T_{1}})] dr\}$.
	\end{center}
	\par
	Проинтегрирую полученное соотношение по объёму $V$, занимаемому смесью, используя \textit{формулу Гаусса-Остроградского}:
	\begin{center}
		$\int_{V} divAdV = \int_{Fs} A^{n} dFs$,
	\end{center}
	где $A$ - вектор; $A^{n}$ - проекция вектора на нормаль к поверхности $S$.
	\small
	\begin{figure}[H]
		\centering
		\begin{tikzpicture}[scale=.5]
			\filldraw[fill=white, thick] (0,0) circle (1.5)
					node[below=9mm, left] at (-1,0) {$V$};
			\node[draw, circle, thick, fill=white, minimum size=1mm, inner sep=0] at (0.3,0.7) {}; 
			\node[draw, circle, thick, fill=black, minimum size=0.5mm, inner sep=0] at (-0.3,-0.7) {}; 
			\node[draw, circle, thick, fill=black, minimum size=0.5mm, inner sep=0] at (-0.3,0.7) {}; 
			\node[draw, circle, thick, fill=black, minimum size=0.5mm, inner sep=0] at (0.9,-0.9) {}; 
			\node[draw, circle, thick, fill=black, minimum size=0.5mm, inner sep=0] at (0.4,0.3) {}; 

			\draw[->, -{Stealth}] (3,0)--(7,0);

			\begin{scope}
				\filldraw[fill=white, very thick] (10,0) circle (1.5)
					node[below=10mm, right] at (10.1,0) {$Fs$};
				\path[clip, preaction={draw, thick}] (10,0) circle (1.5);
				\fill[draw=black, thick, pattern=north west lines] (-12,12)--(-12,-12)--(12,-12)--(12,12) -- cycle;
			\end{scope}
			\draw[latex-] (11.4,0.5)--(13,1) node[above, pos=1.1] () {$A^{n}$};
		\end{tikzpicture}
	\end{figure}
	\large
	\par
	Тогда получаю
	\begin{center}
		$\rho \frac{dS}{dt} = \int_{V} \rho \frac{dS}{dt} dV = \{\int_{Fs} -q_{1}^{n} dFs + \int_{Fs} \frac{\mu_{1k}}{T_{1}} J_{1k}^{n} dFs\} + \int_{V} \{\frac{1}{T_{1}} \tau_{1}^{kl} e_{1}^{kl} + \frac{\vec{q_{1}}}{T_{1}} gradT_{1} + \vec{J_{1k}} grad \frac{\mu_{1k}}{T_{1}} + \int_{0}^{R} q_{12} (\frac{1}{T_{2}} - \frac{1}{T_{1}}) dr + \frac{1}{T_{1}} \int_{0}^{R} \rho_{2}^{0} f r \vec{f_{12}} (\vec{\nu_{1}} - \vec{\nu_{2}}) dr + \int_{0}^{R} \rho_{2}^{0} f \eta [(\frac{\mu_{1k}}{T_{2}} - \frac{\mu_{2k}}{T_{2}}) + i_{1}(\frac{1}{T_{2}} - \frac{1}{T_{1}})] dr\} dV$.
	\end{center}
	\par
	Таким образом, изменение энтропии можно представить в виде суммы двух слагаемых:
	\begin{equation}
		\rho \frac{dS}{dt} = \rho \frac{dS^{(l)}}{dt} + \rho \frac{dS^{(i)}}{dt},
	\end{equation}
	где $\rho \frac{dS^{(l)}}{dt} = -q_{1}^{n} dFs + \int_{Fs} \frac{\mu_{1k}}{T_{1}} J_{1k}^{n} dFs$ отражает изменение энтропии смеси за счёт обмена энергией и массой с окружающей средой. \textit{1-ое слагаемое} характеризует изменение энтропии за счёт обмена энергией ($q_{1}^{n}$ - поток тепла через поверхность), \textit{2-ое слагаемое} - за счёт обмена массой с окружающей средой ($J_{1k}^{n}$ - поток массы через поверхность). \par
	\textit{2-ое слагаемое} $\rho \frac{dS^{(i)}}{dt} = \sigma$ определяет приращение энтропии смеси за счёт внутренних необратимых процессов и представляет собой произведение термодинамическиъ поток на термодинамические движущие силы. \par
	Второй член, связанный с необратимостью процессов, называется \textit{производством} \textit{энтропии} $\sigma$ и представляет собой диссипативную функцию, причём $\sigma \ge 0$. \par
	\textit{Производство энтропии} можно представить в виде
	\begin{center}
		$\sigma = \sum_{i} J_{i} X_{i}$,
	\end{center}
	где $J_{i}$, $X_{i}$ - термодинамические потоки и движущие силы необратимых процессов. \par
	В состоянии равновесия для всех необратимых процессов, происходящих в гетерогенных полидисперсных средах,
	\begin{center}
		$J_{i} = 0$, $\>$, $X_{i} = 0$.
	\end{center}
	Поэтому естественно предположить, что по крайней мере для состояния вблизи равновесия между потоками и движущими силами существуют линейные однородные соотношения в виде
	\begin{center}
		$J_{i} = L_{i} X_{i}$.
	\end{center}
	\begin{center}
		\textbf{Таблица потоков и соответствующих им движущих сил}
	\end{center}

	\small
	\begin{tabularx}{\textwidth}{|*{2}{Y|}}
		\hline
		Потоки (\underline{следствия}) & Силы (\underline{причины}) \\

		\hline
		\multicolumn{2}{|c|}{ТЕНЗОРЫ} \\

		\hline
		$\tau_{1}^{kl}$ - поток вязких напряжений & $e_{1}^{kl}$ - тензор скоростей деформаций \\

		\hline
		\multicolumn{2}{|c|}{ВЕКТОРЫ} \\

		\hline
		$\vec{f}_{12} = L (\vec{\nu}_{1} - \vec{\nu}_{2}) = \mu 3\pi d (\vec{\nu}_{1} - \vec{\nu}_{2})$ (\textit{связь - з. Стокса}) - поток силы взаимодействия между несущей фазой и включением; $\mu$ - вязкость, $3 \pi d $ - размер частицы & $(\vec{\nu_{1}} - \vec{\nu_{2}})$ - движущая сила взаимодействия между потоком и включением \\

		\hline
		$\vec{q}_{1} = \lambda gradT_{1}$ (\textit{связь - з. Фурье})- поток тепла внутри несущей фазы; $\lambda$ - коэффициент теплопередачи & $gradT_{1} = \begin{pmatrix} \frac{\partial T_{1}}{\partial x} \\ \frac{\partial T_{1}}{\partial y} \\ \frac{\partial T_{1}}{\partial z} \end{pmatrix}$ - движущая сила теплопереноса \\

		\hline
		$\vec{J}_{1k} = D grad(\frac{\mu_{1k}}{T_{1}}) = D gradC_{1}$ (\textit{связь - з. Фика}) - поток массы в сплошной фазе & $grad(\frac{\mu_{1k}}{T_{1}}) = gradC_{1k} = \begin{pmatrix} \frac{\partial C_{1k}}{\partial x} \\ \frac{\partial C_{1k}}{\partial y} \\ \frac{\partial C_{1k}}{\partial z} \end{pmatrix}$ - движущая сила массопереноса с сплошной среде \\

		\hline	
		\multicolumn{2}{|c|}{СКАЛЯРЫ} \\
		
		\hline
		$q_{12} = \beta_{12} 4\pi d^{2} (T_{1} - T_{2})$ (\textit{связь - ур. теплопередачи})- поток тепла между несущей фазой и включением; $\beta_{12}$ - коэффициент теплопередачи, $4 \pi d^{2}$ - поверхность частицы & $(T_{1} - T_{2})$ - движущая сила теплопереноса между несущей фазой и включением \\

		\hline
	\end{tabularx}
	\large
\end{document}
