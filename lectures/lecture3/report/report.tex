\documentclass[12pt, a4paper]{report}
\usepackage[top=1cm, left=1cm, right=1cm]{geometry}

\usepackage[utf8]{inputenc}
\usepackage[russian]{babel}

\usepackage{array}
\newcolumntype{M}[1]{>{\centering\arraybackslash}m{#1}}

\usepackage{hyperref}
\hypersetup{
	colorlinks,
	citecolor=black,
	filecolor=black,
	linkcolor=black,
	urlcolor=black
}

\usepackage{sectsty}
\allsectionsfont{\centering}

\usepackage{indentfirst}
\setlength\parindent{24pt}

\usepackage{makecell}

\usepackage{amsmath}

\usepackage{tikz}
\usetikzlibrary{shapes.geometric, arrows.meta, patterns}

\usepackage{listings}
\usepackage{xcolor}
\definecolor{codegreen}{rgb}{0,0.6,0}
\definecolor{codegray}{rgb}{0.5,0.5,0.5}
\definecolor{codepurple}{rgb}{0.58,0,0.82}
\definecolor{backcolour}{rgb}{0.95,0.95,0.92}
\lstdefinestyle{mystyle}{
    backgroundcolor=\color{backcolour},
    commentstyle=\color{codegreen},
    keywordstyle=\color{magenta},
    numberstyle=\normalsize\color{codegray},
    stringstyle=\color{codepurple},
    basicstyle=\ttfamily\footnotesize,
    breakatwhitespace=false,
    breaklines=true,
    captionpos=b,
    keepspaces=true,
    numbers=left,
    numbersep=5pt,
    showspaces=false,
    showstringspaces=false,
    showtabs=false,
    tabsize=2
}

\usepackage{graphicx}
\graphicspath{ {plots/pictures/} }

\usepackage{booktabs}
\usepackage{csvsimple}
\usepackage{longtable}
\usepackage{caption}

\usepackage{pgfplots}
\pgfplotsset{compat=newest, ticks=none}

\usepackage{wrapfig}
\usepackage{float}

\usepackage{multirow, tabularx}
\newcolumntype{Y}{>{\centering\arraybackslash}X}
\renewcommand{\arraystretch}{1.2}

\usepackage{csquotes}

\begin{document}
	\begin{titlepage}
		\begin{center}
			\large \textbf{Министерство науки и высшего образования Российской Федерации} \\
			\large \textbf{Федеральное государственное бюджетное образовательное учреждение высшего образования} \\
			\large \textbf{«Российский химико-технологический университет имени Д.И. Менделеева»} \\

			\vspace*{6cm}
			\LARGE \textbf{ЛЕКЦИЯ №3}

			\vspace*{4cm}
			\begin{flushright}
				\Large
				\begin{tabular}{>{\raggedleft\arraybackslash}p{9cm} p{10cm}}
					Выполнил студент группы КС-36: & Золотухин Андрей Александрович \\
					Ссылка на репозиторий: & https://github.com/ \\
					& CorgiPuppy/ \\
					& info-proccesses-systems-theory-labs \\
				\end{tabular}
			\end{flushright}

			\vspace*{5cm}
			\Large \textbf{Москва \\ 2025}
		\end{center}
	\end{titlepage}

	\tableofcontents
	\thispagestyle{empty}
	\newpage

	\pagenumbering{arabic}

	\section*{Термодинамический анализ}
	\addcontentsline{toc}{section}{Термодинамический анализ}

	\subsection*{Термодинамика линейных необратимых систем}
	\addcontentsline{toc}{subsection}{Термодинамика линейных необратимых систем}
	\large
	В термодинамике необратимых процессов \textit{Ларс Онзагер} сформулировал:
	\begin{displayquote}
		При небольших отклонениях от равновесия термодиначеский поток можно представить в виде линейной комбинации термодинамических потоков и термодинамических движущих сил:
		\begin{center}
			$J_{i} = L_{i1}X_{1} + L_{i2}X_{2} + L_{i3}X_{3} + \ldots + L_{ij}X_{j} + \ldots + L_{iN}X_{N}$,
		\end{center}
		где $X_{i}$ - движущая сила, сопряжённая с потоком $J_{1k}$ определяет прямой эффект; $X_{j}$ - движущая сила, характеризующая перекрёстный эффект; $L_{1k}$ - феноменологические эффекты: $L_{ii}$ - при прямом эффекте, $L_{ij} (i \neq j)$ - при перекрёстном эффекте.
	\end{displayquote}
	\par
	Таким образом, диссипативную функцию (\textit{производство энтропии}) можно представить в виде квадратичной положительно определённой формы:
	\begin{center}
		$\sigma = \sum_{i=1}^{n} J_{i} X_{i} = \sum_{i=1}^{n} (\sum_{j=1}^{n} L_{ij} X_{j}) X_{i} \ge 0$.
	\end{center}
	\par
	Положительность квадратичной формы следует из \textit{второго закона термодинамики}. В изолированной системе энтропия системы не может убывать. Поэтому если обмен энергией и массой с внешней средой отсутствует, изменение энтропии системы равно производству энтропии:
	\begin{center}
		$\rho \frac{dS}{dt} = \rho \frac{dS^{(i)}}{dt} = \sigma \ge 0$
	\end{center}
	и, следовательно, производство энтропии неотрицательно. \par
\end{document}
